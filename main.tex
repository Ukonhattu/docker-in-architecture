\documentclass[fleqn,12pt]{olplainarticle}
% Use option lineno for line numbers 
\renewcommand{\baselinestretch}{1.3}

\title{Docker in Software Architecture}

\author[1]{Daniel Koch}
\author[2]{Sebastian Sergelius}
\affil[1]{daniel.koch@helsinki.fi}
\affil[2]{github.com/sebazai}

\keywords{docker, software architecture, containers}

\begin{abstract}
Please provide an abstract of no more than 300 words. Your abstract should explain the main contributions of your article, and should not contain any material that is not included in the main text. 
\end{abstract}

\begin{document}
\flushbottom
\maketitle
\thispagestyle{empty}

\section*{Introduction}

With the rise of DevOps and the use of Docker, containerization has become more and more relevant to think about even
in software architecture. DevOps does not have any official definition.
One formal definition is given by \cite{Jabbari_devops}: 
\begin{displayquote}
"DevOps is a development methodology aimed at bridging the gap between Development and Operations, emphasizing communication and collaboration, continuous integration, quality assurance and delivery with automated deployment utilizing a set of development practices".
\end{displayquote}
Another way to put DevOps is that it consists of the development of software and operations and that DevOps means that development, release, configuration, and monitoring are all done by the same people, rather than having separate teams for every part \citep{hy:DevOps_with_Docker}.

Containerization is commonly


\section*{overview of the main functional features (what does it do for you?) if applicable}

\subsection*{Docker Daemon}

\subsection*{Docker Client}

\subsection*{Docker Images}

\subsection*{Docker Containers}


\section*{A technical overview (what is it?)}

\subsection*{Docker Architecture}
Techical overview

Kelasin, että kirjoitetaan tänne vähän tosta mitä se vaatii namespacea cgrouppia jne ja sit vois niiden perusteella analysoida noita quality attributeja? En tiedä. Mitä mieltä olet? On jotenkin vaikea kirjoittaa -ilities, ilman että käy läpi tuon techin, enkä kyllä löydä hyviä lähteitäkään. 

\section*{ Quality Attributes}

The qualities (-ilities) that the TUS claims to support (what?) and justifications for the claims (how?)
Based on how Docker is built, the system has some qualities hoisted in it. In this section we will 

\subsection*{Portability}
One key aspects of Docker is its Portability. With software portability we refer to the usability of the same software in different environments, such as the underlying operating system or hardware \citep{wiki:Software_portability}. Portability is achieved based on how the Docker technology is built. Each container includes all the dependencies required for the application to work and only requires an underlying operating system and infrastructure that supports the Docker Engine \citep{hy:DevOps_with_Docker}.
\subsection*{Security}
\subsection*{Scalability}
\subsection*{Maintainability}


\section*{What is it not suited for (any inherent limitations or reported issues) - may be challenging to research}

Perhaps

\section*{Architectural impact - what aspect of application/system architecture is affected by using it and how?}
yes
\section*{Tool support / automation provided (can be also from 3rd parties) to help application developers be more productive}

\section*{Acknowledgments}

Additional information can be given in the template, such as to not include funder information in the acknowledgments section.

\bibliography{sample}

\end{document}