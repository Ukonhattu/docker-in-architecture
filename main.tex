\documentclass[fleqn,10pt]{olplainarticle}
% Use option lineno for line numbers 

\title{Docker in Software Architecture}

\author[1]{Daniel Koch}
\author[2]{Sebastian Sergelius}
\affil[1]{Address of first author}
\affil[2]{Address of second author}

\keywords{docker, software architecture, containers}

\begin{abstract}
Please provide an abstract of no more than 300 words. Your abstract should explain the main contributions of your article, and should not contain any material that is not included in the main text. 
\end{abstract}

\begin{document}

\flushbottom
\maketitle
\thispagestyle{empty}

\section*{Introduction}

What is docker? What is it used for? What are containers and why use them? Maybe something about how popular docker has become. Does docker affect architecture?
The purpose and the domain (what to use it for?);

\section*{Methods and Materials}

Guidelines can be included for standard research article sections, such as this one.

\section*{ overview of the main functional features (what does it do for you?) if applicable}
\label{sec:examples}
 overview of the main functional features (what does it do for you?) if applicable

\subsection*{A technical overview (what is it?) }

Techical overview

\section*{The qualities (-ilities) that the TUS claims to support (what?) and justifications for the claims (how?)}

Explain something

\section*{What is it not suited for (any inherent limitations or reported issues) - may be challenging to research}

Perhaps

\section*{Architectural impact - what aspect of application/system architecture is affected by using it and how?}
yes
\section*{Tool support / automation provided (can be also from 3rd parties) to help application developers be more productive}

\section*{Acknowledgments}

Additional information can be given in the template, such as to not include funder information in the acknowledgments section.

\bibliography{sample}

\end{document}